\documentstyle{jarticle}
%  議事録スタイル(meeting.sty)の使い方 サンプル    980514%  
%  ・細かいことはまだ不明ですが、大体のところは使えるようになりました
%  ・図や表はもちろん箇条書きなどの環境も使えます
%  ・複数ページに及ぶときの処理もきれい
%  ・議事録の主要なヘッダーはプリアンブルで指定します。
%
%
%\where{第2会議室}            
% 会議場所\when{\today}                
% 会議日時\who{おれ,おまえ,その他}     
% 出席者名(半角コンマで区切る)\date{\today}                
% 議事録作成日\回覧{係長,課長,部長,社長}   
% 回覧先(半角コンマで区切る)\配布{Bill Clinton,橋本龍太郎}     
% 配布先(半角コンマで区切る)\西暦                        
% 日付を西暦とする場合に指定

\title{ 議事録:第一回情報工学特別演習} % 文書のタイトル
\author{応用数理研究室 平岡 蒼汰} % 著者
\date{$2019$年$4$月$16$日(火)$23:00$} % 日付

\begin{document}
\maketitle % 表題の出力
%
% 同一文書内に二つ目以降の \議事録{} があると、そこで改ページされ、
% 次ページから新たな議事録が始まります。議事録ヘッダーも新たに
% 書かれます。
\section{基礎事項}
\subsection{会議名}
\noindent 「情報工学特別演習ガイダンス」
\subsection{開催日時}
\noindent $2019$年$4$月$16$日(火)$5$限$(16:20 \sim 17:50)$
\subsection{場所}
\noindent メディアセンター$2$階第$1$セミナー室
\subsection{参加者(敬称略)}
\noindent 指導教員:阿萬、岡野、伊藤、安藤、森岡\\
学生:安里、宮本、山中、渡邊、長澤、平岡
\section{会議内容}
\subsection{前回からの宿題}
\noindent 特になし。
\subsection{議論した内容}
\noindent 情報工学特別演習の授業の進め方\\
情報工学特別演習の授業の注意事項
\subsection{結論}
授業について
\begin{itemize}
  \item 授業回数は$30$回(前期後期合わせて)
 \item 各授業内容:システムの共同開発
  \item 学期末には報告会(進歩状況のプレゼンテーション
  \item 報告会の日程調整は$2$週間以上前に連絡
  \item プレゼンテーションは全員参加
 \item 担当者を決定してシステム開発を行う
\end{itemize}

授業の進め方\\
授業時間:火曜日$5$限$(16:20 \sim 17:50)$時間変更をしたい場合は随時連絡
議事録を作成する人が授業場所を決定し、連絡を行う。\\
議事録の作成順番:平岡$\rightarrow$長澤$\rightarrow$宮本$\rightarrow$安里$\rightarrow$山中$\rightarrow$渡邊\\
最初に共通のバージョン管理システムの勉強を行う。\\
$\rightarrow$Git,GitHubの練習\\
作成するシステムの内容を決定する\\
$\rightarrow$例)飲み会の日程調整など\\

\subsection{次回までの課題}
\noindent Git,GitHubの基礎知識の習得(担当:全員)\\
開発システム案の作成(担当:全員)\\
授業場所の決定(担当:平岡)

\section{次回の授業}
\noindent 日時:$4$月$23$日(火)$5$限$(16:20 \sim 17:50)$\\
場所:未定

\end{document}